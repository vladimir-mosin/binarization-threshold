\section{Введение}
Прогнозирование востребованности контента на основе моделей машинного обучения имеет высокую актуальность в современных медиа, и этому  есть несколько важных причин.
Прежде всего, это увеличение конкуренции (см. [8]). С ростом количества контента, конкуренция за привлечение внимания аудитории становится все более жесткой. Точное прогнозирование востребованности контента позволяет создавать более привлекательный и интересный контент, который может более эффективно привлекать и удерживать пользователей.

Кроме того, прогнозирование востребованности контента позволяет оптимизировать бюджет и ресурсы, распределяя их наиболее эффективно. Зная, какой контент будет популярным, можно сосредоточиться на его производстве и продвижении, тем самым снизив затраты на контент, который может не найти достаточный отклик у аудитории.

Наконец, прогнозирование позволяет анализировать текущие тренды и предсказывать будущие. Это может быть полезным для создания нового контента, который будет актуальным в будущем, и для прогнозирования развития рынка контента в целом.
Все эти факторы делают задачу прогнозирования востребованности контента на основе моделей машинного обучения весьма важной для создателей контента, медиа- и развлекательных компаний, маркетологов и пользователей (см. [2]). 

\subsection{Теоретическая часть}
Задача бинарной классификации в машинном обучении заключается в отнесении объектов к одной из двух возможных категорий (см. [9], [10]). Это означает, что модель обучается разделить данные на два класса, условно положительные и отрицательные. В медиа-индустрии широкое распространение получила задача разделения позитивного или негативного отношения к тексту, комментарию, видеоролику и~т.~д. 

Важной частью задачи бинарной классификации является оценка качества модели, которая может выполняться с помощью метрик, таких как точность, полнота, F-мера и мера AUC. Оценка позволяет измерить, насколько точно модель предсказывает классы, и насколько хорошо она может обобщать на новые данные. 

В нашей работе мы используем метрику AUC, которая является одной из самых важных и информативных метрик в задаче бинарной классификации (см. [4], [5], [7]). 

\subsection{Постановка задачи}
Имеются данные об объектах, в качестве которых выступают видеоролики,  размещенные на канале одного из ведущих хостингов. Требуется 1)~построить бинарную модель классификации, прогнозирующую востребованность контента среди посетителей и зрителей канала; 2)~пользуясь метрикой AUC, оценить эффективность модели; 3)~повысить ее эффективность путем определения оптимального порога бинаризации.

\subsection{Библиотеки} 
Для проведения всех расчетов и визуализации результатов мы используем среду \texttt{Jupyter Notebook}, язык программирования \texttt{Python} и его библиотеки: \texttt{pandas}, \texttt{sklearn}, \texttt{matplotlib}. 