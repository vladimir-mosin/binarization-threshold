\section{Результаты}
	
	Напомним, что в качестве показателя востребованности контента мы взяли бинарный признак 'Доля подписок', принимающий два возможных значения: 'Низкая', 'Высокая'. Используя это признак в качестве целевой функции, мы построили модель классификации объектов (видеороликов канала), позволяющую прогнозировать востребованность объектов, то есть, относить новый, не имеющий априорной классификации объект к одному из двух типов: a)~объект с низкой долей подписок, b)~объект с высокой долей подписок. 
	
	Целью нашей работы было получение бинарного прогноза из вероятностного путем выбора оптимального порога бинаризации и сравнение оптимизированного прогноза с бинарным дефолтным прогнозом. В итоге мы продемонстрировали достоверный сдвиг в сторону увеличения метрики AUC при переходе от дефолтного к оптимизированному прогнозу.