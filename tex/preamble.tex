\documentclass[a4paper,12pt]{article}

\usepackage[russian]{babel}
\usepackage{cmap}
\usepackage[utf8]{inputenc}
\usepackage[usenames]{color}
\usepackage{tabularray}
\usepackage{xcolor}
\usepackage{graphicx} 
\usepackage{subfigure}
\usepackage{subcaption}

\usepackage[unicode]{hyperref} % цвета гиперссылок
\hypersetup{
	colorlinks,
	citecolor=black,
	filecolor=black,
	linkcolor=blue,
	urlcolor=black
}

\usepackage{geometry} % задаёт поля 
%\geometry{left=3cm}
%\geometry{right= 1.5cm}
%\geometry{top=2cm}
%\geometry{bottom=2cm} 

\usepackage{enumitem} % настраивает работу со списками:
\def\labelitemi{—} % ... задаёт длинное тире как стандартный маркер ненумерованного списка
\setlist{nolistsep} %  ... убирает дополнительный отступы между элементами списка


% удаляет названия и продолжение следует и т. для таблиц, будет только таблица без всего
\DefTblrTemplate{contfoot-text}{default}{}
\DefTblrTemplate{conthead-text}{default}{}
\DefTblrTemplate{caption}{default}{}
\DefTblrTemplate{conthead}{default}{}
\DefTblrTemplate{capcont}{default}{}


\title{Бинарная модель востребованности контента\\и ее вероятностный прогноз}
\author{В. Г. Мосин}
\date{}

