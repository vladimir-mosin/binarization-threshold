\section{Литература}

\begin{enumerate}
	\item Хейдт М. Изучаем Pandas / М. Хейдт;~--- Москва: ДМК Пресс, 2018.~--- 438 с.
	\item Бурков А. Машинное обучение без лишних слов / А.~Бурков;~--- СПб: Питер, 2020.~--- 192 с.
	\item Николенко С. Глубокое обучение. Погружение в мир нейронных сетей / С.~Николенко, А.~Кадурин, Е.~Архангельская;~--- СПб: Питер, 2018.~--- 481 с.
	\item Лимановская, О.В. Основы машинного обучения : учебное пособие / О.В.~Лимановская, Т.И.~Алферьева;~--- Екатеринбург : Изд-во Урал. ун-та, 2020.~--- 88 с.
	\item Шолле, Ф. Глубокое обучение на Python / Ф. Шолле;~--- СПб.: Питер, 2018.~--- 400 с.
	\item Вьюгин, В. В. Математические основы теории машинного обучения и прогнозирования / В. В. Вьюгин;~--- М.: МЦИМО.~---  2013.~--- 387~с.
	\item Бринк Х. Машинное обучение / Х. Бринк, Дж. Ричардс, М. Феверолф~--- СПб.: Питер, 2017.~--- 336 с.
	\item Дьяконов А. Г. Прогноз поведения клиентов супермаркетов с помощью весовых схем оценок вероятностей и плотностей / А.~Г.~Дьяконов // Бизнес-информатика.~--- 2014. Т. 1, № 27.~--- С. 68–-77.
	\item Михеев, А. В. Решение задач классификации методами машинного обучения / А. В. Михеев // Молодой ученый.~--- 2021.~--- № 21 (363).~--- С. 107--110. 
	\item Неделько В. М. Исследование эффективности некоторых линейных методов классификации на модельных распределениях // В. М. Не\-делько // Машинное обучение и анализ данных.~--- 2016. Т. 2, №3.~--- С. 305–-329. 
\end{enumerate}
